%%%%%%%%%%%%%%%%%%%%%%%%%%%%%%%%%%%%%%%%%%%%%%%%%%%%%%%%%%%%%%%%%%%%%%%%

%%% LaTeX Template for AAMAS-2026 (based on sample-sigconf.tex)
%%% Prepared by the AAMAS-2026 Publication Chairs based on the version from AAMAS-2025. 

%%%%%%%%%%%%%%%%%%%%%%%%%%%%%%%%%%%%%%%%%%%%%%%%%%%%%%%%%%%%%%%%%%%%%%%%

%%% Start your document with the \documentclass command.


%%% == IMPORTANT ==
%%% Use the first variant below for the final paper (including author information).
%%% Use the second variant below to anonymize your submission (no author information shown).
%%% For further information on anonymity and double-blind reviewing, 
%%% please consult the call for paper information
%%% https://cyprusconferences.org/aamas2026/submission-instructions/

%%%% For anonymized submission, use this
\documentclass[sigconf,anonymous]{aamas} 

%%%% For camera-ready, use this
%\documentclass[sigconf]{aamas} 


%%% Load required packages here (note that many are included already).

\usepackage{balance} % for balancing columns on the final page
\usepackage{acronym}

%%%%%%%%%%%%%%%%%%%%%%%%%%%%%%%%%%%%%%%%%%%%%%%%%%%%%%%%%%%%%%%%%%%%%%%%

%%% AAMAS-2026 copyright block (do not change!)

\setcopyright{ifaamas}
\acmConference[AAMAS '26]{Proc.\@ of the 25th International Conference
on Autonomous Agents and Multiagent Systems (AAMAS 2026)}{May 25 -- 29, 2026}
{Paphos, Cyprus}{C.~Amato, L.~Dennis, V.~Mascardi, J.~Thangarajah (eds.)}
\copyrightyear{2026}
\acmYear{2026}
\acmDOI{}
\acmPrice{}
\acmISBN{}


%%%%%%%%%%%%%%%%%%%%%%%%%%%%%%%%%%%%%%%%%%%%%%%%%%%%%%%%%%%%%%%%%%%%%%%%

%%% == IMPORTANT ==
%%% Use this command to specify your submission number.
%%% In anonymous mode, it will be printed on the first page.

\acmSubmissionID{<<submission id>>}

%%% Use this command to specify the title of your paper.

\title[Collective Behaviour via Meso-Macro Design]{Collective Behaviour-based Design through Meso- and Macro-Descriptions}

%%% Provide names, affiliations, and email addresses for all authors.

\author{Arthur Pendragon}
\affiliation{
  \institution{Camelot Castle}
  \city{Camelot}
  \country{United Kingdom}}
\email{king.arthur@camelot.uk}

\author{Nimue}
\affiliation{
  \institution{The Lady's Lake}
  \city{Avalon}
  \country{United Kingdom}}
\email{lady.of.the.lake@avalon.uk}

%%% Use this environment to specify a short abstract for your paper.

\begin{abstract}
\meta{TODO}
\end{abstract}

%%% Use this command to specify a few keywords describing your work.
%%% Keywords should be separated by commas.

\keywords{Collective behaviour, Macroprogramming, Multi-agent systems}

%%%%%%%%%%%%%%%%%%%%%%%%%%%%%%%%%%%%%%%%%%%%%%%%%%%%%%%%%%%%%%%%%%%%%%%%

%%% Include any author-defined commands here.
         
\newcommand{\BibTeX}{\rm B\kern-.05em{\sc i\kern-.025em b}\kern-.08em\TeX}

%%%%%%%%%%%%%%%%%%%%%%%%%%%%%%%%%%%%%%%%%%%%%%%%%%%%%%%%%%%%%%%%%%%%%%%%

\newcommand{\meta}[1]{{\color{blue}#1}}

\acrodef{ai}[AI]{artificial intelligence}
\acrodef{api}[API]{application program interface}
\acrodef{bpmn}[BPMN]{business process modelling notation}
\acrodef{ccps}[CCPE]{collective cyber-physical ecosystem}
\acrodef{CDN}{content delivery network}
\acrodef{cas}[CAS]{collective adaptive system}
\acrodef{ci}[CI]{collective intelligence}
\acrodef{cci}[CCI]{computational collective intelligence}
\acrodef{CLR}{common language runtime}
\acrodef{scps}[sCPS]{smart cyber-physical system}
\acrodef{cps}[CPS]{cyber-physical system}
\acrodef{CS}{collective system}
\acrodef{DSL}{domain-specific language}
\acrodef{DNS}{domain name system}
\acrodef{ECC}{edge-cloud continuum}
\acrodef{GPL}{general-purpose language}
\acrodef{hmi}[HMI]{human-machine interface}
\acrodef{iot}[IoT]{Internet of Things}
\acrodef{mas}[MAS]{multi-agent system}
\acrodef{JVM}{Java Virtual Machine}
\acrodef{se}[SE]{software engineering}
\acrodef{marl}[MARL]{multi-agent reinforcement learning}
\acrodef{RL}{reinforcement learning}
\acrodef{rv}[RV]{runtime verification}
\acrodef{VM}{virtual machine}
\acrodef{DSE}{design space exploration}
\acrodef{MAPE-K}{Monitor-Analyse-Plan-Execute-Knowledge}

\begin{document}

%%% The following commands remove the headers in your paper. For final 
%%% papers, these will be inserted during the pagination process.

\pagestyle{fancy}
\fancyhead{}

%%% The next command prints the information defined in the preamble.

\maketitle 

%%%%%%%%%%%%%%%%%%%%%%%%%%%%%%%%%%%%%%%%%%%%%%%%%%%%%%%%%%%%%%%%%%%%%%%%


\meta{
\begin{itemize}
\item \url{https://cyprusconferences.org/aamas2026/call-for-blue-sky-ideas/}
\item DEADLINE: 2025-12-03
\item TRACK GOALS: provoke visionary ideas, long-term challenges, new research opportunities, and debate; focus on novel, overlooked, or under-represented application areas to which agent research may contribute
\item EVAL CRITERIA: 
(i) Relevance and Advancement (future-thinking);
%     Are the ideas presented of interest to members of the Agents and MAS research communities?
%    Do the ideas push forward the envelope of Agents/MAS application into the future?

(ii) Novelty and Vision (``Blue Sky''-ness);
%     How much the ideas are visionary (vs. next step w.r.t. state of the art), novel, or out of the box.

(iii) Background and Foundation ( grounded-ness)
%     Are the ideas motivated and grounded in a solid understanding of the existing state-of-the-art and older foundational research/theory?
%    Do the ideas consider relatively new research ideas (e.g. published in the last 3 years)?
(iv) Impact and Practicality (real-world applicability)
%     How deep and wide is the envisioned impact of the ideas on the research agenda in the Agents and MAS communities?
%    How deep and wide is the envisioned impact on the intended beneficiaries of the ideas presented, including practical considerations?

(v) Rigour and Clarity (strength of contribution)
%     Are the ideas presented in a clear and rigorous way?
%    How strong is the level of critical reflection applied in the exploration of these ideas?
\item LIMITS: 4 pages (+ refs), DOUBLE BLIND
\end{itemize}
}


\section{Introduction}
\label{sec:intro}

\paragraph{Long-standing challenge: engineering collective behaviour.} Effectively engineering the collective behaviour 
  that has to be exhibited 
  by a group or multitude of 
  interacting agents
  is a long-standing challenge~\cite{readings-dai-1988,DBLP:journals/swarm/BrambillaFBD13,DBLP:journals/sttt/NicolaJW20,DBLP:journals/tosem/CasadeiAADPSTV25}.
%
This has historically been addressed  
  in research areas 
  like
  distributed artificial intelligence~\cite{readings-dai-1988}, 
  swarm robotics~\cite{DBLP:journals/swarm/BrambillaFBD13},
  and collective adaptive systems~\cite{DBLP:journals/sttt/NicolaJW20}.

The essence of the issue lies in \emph{complexity} of system dynamics 
  due to the potentially huge number of degrees of freedoms
  and system trajectories.
%
The design activity normally focusses on \emph{individuals' behaviours}
  and their \emph{coordination/interaction},
  which normally lead to \emph{emergent} phenomena.
%
Therefore, the approach is subject to the \emph{local-to-global} or \emph{micro-to-macro} prediction problem,
  generally addressed by (often expensive) \emph{simulation}
  of system behaviour in a range of environmental settings.
%
Evolutionary~\cite{} and \ac{marl}~\cite{Hou2025dml-swarm} approaches 
  have contributed to this quest
  by usign supervisory signals to search through the policy space, but continue to suffer from scalability issues~\cite{dingbang2024scalability-marl}.


\cite{DBLP:journals/csur/Casadei23}

\section{Background on Collective Behaviour Engineering}

\section{Meso- and Macro-programming for Multi-Agent Systems}

\section{Research Opportunities and Challenges}



\bibliographystyle{ACM-Reference-Format} 
\bibliography{biblio}

%%%%%%%%%%%%%%%%%%%%%%%%%%%%%%%%%%%%%%%%%%%%%%%%%%%%%%%%%%%%%%%%%%%%%%%%

\end{document}

%%%%%%%%%%%%%%%%%%%%%%%%%%%%%%%%%%%%%%%%%%%%%%%%%%%%%%%%%%%%%%%%%%%%%%%%

